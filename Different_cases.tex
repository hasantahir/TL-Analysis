\documentclass[11pt]{article}
% Horizontal Magnetic Dipole over a lossy half-space
\usepackage[utf8]{inputenc} % Use it to include other characters than ABC
\usepackage[cmex10]{amsmath}
\usepackage{mdwmath}
\usepackage{mdwtab}
\usepackage{hyperref}
\usepackage{physics} % For using the oridnary derivative nomenclature
\usepackage{datetime} % Insert date and time
\usepackage[letterpaper, margin=1in]{geometry}
\usepackage{graphicx}
% \usepackage{mathptmx} % Times new Roman

% ------------------------------- Useful Tricks Learnt
% Use ={}& to align subequations to the left
% Use = for single equations
%
% ----------------- To compile with references use the following order in Shell"
% 1. pdflatex filename.tex
% 2. bibtex filename (no extension)
% 3. bibtex filename (no extension)
% 4. pdflatex filename.tex
% -----------------

\begin{document}
  \title{\textsc{Equivalent Tranmission Line Models for Layered Structures with Sources}\\}
  \date{\footnote{Last Modified: \currenttime, \today.}}
  \maketitle
 We consider a multilayered structure with piece-wise material that is assumed unbounded in the tranverse direction
  \begin{subequations}
    \begin{align}
      \nabla\times{\bf E} ={}& -j \omega \mu \bf{H} -\bf{M},
  		\label{eq:E}\\
      \nabla\times{\bf H} ={}& j \omega \varepsilon \bf{E} + \bf{J}.
  		\label{eq:H}
    \end{align}
    \label{eq:MaxE}
  \end{subequations}

For boundary-value problems involving planarly multilayer structures displaying symmetry along the $z$ direction, it is desirable to decompose the $\mathbf{\nabla}$ operator into two components, one $\dv{}{z}$ and the other to a transverse (to z) operator, $\mathbf{\nabla_t}$ \cite[p. 64]{felsen1994radiation}. The analysis can be simplified by taking Fourier transform represented by the operator $\mathcal{F}$, in both $x$ and $y$ directions. This transforms the $\mathbf{\nabla}$ operator to $-j k_x \mathbf{\hat{x}} - j k_y \mathbf{\hat{y}} + \mathbf{\hat{z}}\dv{}{z}$ containing only a single derivative in $z$.
The Fourier transform along with its inverse is defined as:

\begin{subequations}
  \begin{align}
    \mathcal{F}[f(\mathbf{r})] \equiv \tilde{f}(\mathbf{k_{\rho}},z) ={}& \int_{-\infty}^{\infty} \int_{-\infty}^{\infty}
    f(\mathbf{r}) \exp(-j \mathbf{k_{\rho}} \cdot \mathbf{\rho}) dx dy
    \label{eq:Fourier}\\
    \mathcal{F}^{-1}[\tilde{f}(\mathbf{k_{\rho}},z)] \equiv f(\mathbf{r}) ={}& \frac{1}{(2\pi)^2} \int_{-\infty}^{\infty} \int_{-\infty}^{\infty} \tilde{f}(\mathbf{k_{\rho}},z)
    \exp(j \mathbf{k_{\rho}} \cdot \mathbf{\rho}) dk_x dk_y
    \label{eq:IFourier}
  \end{align}
  \label{eq:FT}
\end{subequations}
 where,
 \begin{equation}
   \mathbf{\rho} = x\mathbf{\hat{x}} + y\mathbf{\hat{y}},
   \mathbf{k_{\rho}} = k_x\mathbf{\hat{x}} + k_y\mathbf{\hat{y}},
 \end{equation}

The Maxwell's equations (\ref{eq:MaxE})


\bibliography{mylib}
\bibliographystyle{ieeetr}

\end{document}
